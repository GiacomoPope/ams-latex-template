\documentclass[10pt]{article}

% Use Minion Pro
% This can be a pain to set up...
\usepackage{fontspec}
\setmainfont{Minion Pro}
\usepackage[minionint, onlymath, mathlf]{MinionPro}
\usepackage{amsthm}

% Used for enforcing upper/lowercase
\usepackage{textcase}

% line spacing
\usepackage{setspace}

% paper size
\usepackage[paperheight=234mm,
        	paperwidth=156mm,
        	left=1.02in,
            right=1.02in,
            top=0.8in,
            bottom=1.2in,
            headsep=0.325in,
            footskip=0.2in,
            ]{geometry}

% Set Title format
\makeatletter
\def\@maketitle{%
  \newpage
  \null
  \vskip 4.75em%
  \let\footnote\thanks
  \begin{center}%
	{\setstretch{0.65} \Large \scshape {\@title} \par}
    \vskip 0.4em%
    \textsc{by} 
    \vskip -0.1em%
    {
      \lineskip .5em%
      \begin{tabular}[t]{c}%
        \small \MakeTextUppercase{\@author}
      \end{tabular}\par}%
    \vskip 0em%
  \end{center}%
  \par
  \vskip -0.5em}
\makeatother

% symbols for footnote
\usepackage[symbol*]{footmisc}
\DefineFNsymbolsTM{myfnsymbols}{% def. from footmisc.sty "bringhurst" symbols
  \textasteriskcentered *
  \textdagger    \dagger
  \textdaggerdbl \ddagger
  \textsection   \mathsection
  \textbardbl    \|%
  \textparagraph \mathparagraph
}%
\setfnsymbol{myfnsymbols}
% The title is typed all lowercase as 
% \MakeTextLowercase also effects the
% footnote and I've wasted enough time
% already so I'm not fixing this...
\title{the structure of the number of \\ representations function \\ in a binary quadratic \\ form\protect\footnote{Presented to the Society, October 29, 1932; received by the editors September 26,1932.}}
\author{Gordon Pall}

\begin{document}
% Set footnote += 1 to account
% for the footnote in the title
\setcounter{footnote}{1}
% Set the page number to match the journal
\setcounter{page}{491}

\maketitle

This paper contains, primarily, the extension to any integral, binary quadratic form of the results of a recent article\footnote{Mathematische Zeitschrift, vol. 36 (1933); this article will be referred to here as MZ.} concerning positive, binary, quadratic forms. With suitable conventions almost all the results carry over without change, though some of the proofs need slight alterations. Incidentally, there are treated automorphs of binary quadratic forms, and (rather fully) \; properties \hspace{1pt} of \hspace{1pt} sets \hspace{1pt} of \hspace{1pt} representations \hspace{1pt} (representations \hspace{1pt} equivalent \hspace{1pt} through automorphic transformations) in a binary quadratic form.

1. Dirichlet\footnote{Cf. ��86and 87 of \emph{Vorlesungen\"uber Zahlentheorie}, 4th edition, 1894.} has already in all essentials extended the notion of number of representations to indefinite forms. We shall utilize the following equivalent definition.

Two representations $(x,y)$ and $(x^\prime, y^\prime)$ of m in the form $f= [a, b, c ],$\footnote{We use Kronecker forms, for simplicity, throughout. Hence $[a, b, c]$ stands for $ax^2 + bxy + cy^2$. For the automorphs see �2.} that is, two integral solutions of

\begin{equation}
\label{eq:bqf}
ax^2 + b xy + c y^2 = m,	
\end{equation}
will be called \emph{equivalent} if they are transformable one into the other by integral automorphs of $f$. The class of all representations equivalent to a given one will be called a \emph{set} of representations. The number of sets of representations of $m$ in $f$ will be denoted by $f(m)$. (In MZ, $f(m)$ denoted the number of representations of $m$ in $f$.)

This definition becomes more interesting when we observe that, if $d( =b^2-4ac) > 0$, the number of sets of representations of $m$ in $[a, b, c]$ is equal to the actual number of solutions of (\ref{eq:bqf}) together with certain inequalities (cf. Theorem 9). The writer developed the theory of these inequalities before noticing that Dirichlet (�87, loc. cit.) obtains one such system. However, we shall obtain a substantial improvement on Dirichlet's inequalities and give a more complete discussion of the infinitely many alternative systems. The treatment in �3 is fairly comprehensive.
\end{document}
